\documentclass[12pt, letterpaper]{article}
\usepackage[utf8]{inputenc}
\usepackage{graphicx}
\graphicspath{ {C:/Users/Martin/Pictures/} }

\renewcommand*\contentsname{Indholdsfortegnelse}

\begin{document}

\begin{titlepage}

\newcommand{\HRule}{\rule{\linewidth}{0.5mm}} % Defines a new command for the horizontal lines, change thickness here

\center % Center everything on the page
 
%----------------------------------------------------------------------------------------
%	HEADING SECTIONS
%----------------------------------------------------------------------------------------

\textsc{\LARGE Aarhus universitet}\\[1.5cm] % Name of your university/college
\textsc{\Large DSB}\\[0.5cm] % Major heading such as course name
\textsc{\large Semester 3}\\[0.5cm] % Minor heading such as course title

%----------------------------------------------------------------------------------------
%	TITLE SECTION
%----------------------------------------------------------------------------------------

\HRule \\[0.4cm]
{ \huge \bfseries Mini-projekt}\\[0.4cm] % Title of your document
\HRule \\[1.5cm]
 
%----------------------------------------------------------------------------------------
%	AUTHOR SECTION
%----------------------------------------------------------------------------------------

% If you don't want a supervisor, uncomment the two lines below and remove the section above
\Large \emph{Studerende:}\\[1cm]
Mette \textsc{Hammer Nielsen-Kudsk}\\[0,5cm] % Your name
Martin \textsc{Banasik}\\[0,5cm] % Your name
Finja \textsc{Jette Ralfs}\\[0,5cm] % Your name
%----------------------------------------------------------------------------------------
%	DATE SECTION
%----------------------------------------------------------------------------------------

{\large \today}\\[1,2cm] % Date, change the \today to a set date if you want to be precise

%----------------------------------------------------------------------------------------
%	LOGO SECTION
%----------------------------------------------------------------------------------------

\includegraphics[scale=0.5]{billeder/au}\\ % Include a department/university logo - this will require the graphicx package
 
 %\includegraphics[width=0.6\textwidth]{figurer/ASE}~\\[1cm]
%----------------------------------------------------------------------------------------

\vfill % Fill the rest of the page with whitespace


\end{titlepage}

\tableofcontents
\newpage

\section*{Indledning}

Når vi går fra analoge signaler til digitale, så finder vi repræsentationer af det kontinuerer signal. Dette kalder vi samples og betegnes med N.
Når vi har flere samples på et signal, betegnes intervallet i mellem samples som $T_s$, samplingstid. Når vi har samplingstid kan vi indføre samlingsfrekvens, det inverse af samlingstid. $$f_s = 1/T_s$$
Så snart at vi har $T_s$, ved vi at vi har med et digitalt signal at gøre.


Ved opsætning af sampletidsaksen, definerer vi først vores sampletæller, $n$:
$$n = [0:N-1]$$ Hvor N er antal samples.
Efterfølgende bestemmer vi vores sampletidspunkter, $t$:
$$t = n*T_s$$
Vi kan nu indføre:
$$x(t_s) = X(n*T_s) = X(n)$$
Vi har altid en grundfrekvens og den kalder vi altid $f_0$.

\section*{Aliasering}

Alias = et andet navn for noget/tvetydighed.
Vi har tre forskellige slags alias:
\begin{itemize}
\item Forkert samling - både for mange samples og for få
\item Gentagelser
\item Spejling
\end{itemize}

Shannons sandheds sætning
$$Indsæt formel$$
I praksis er dette aldrig lig med, men skal altid overholdes. 
Nyquist-frekvens:
$$f_nyquist = f_s/2$$
Altså definereret som halvdelen af samlingsfrekvensen, $$f_s$$


%\begin{center}
%\includegraphics[width=\textwidth]{billeder/download}
%\end{center}



\section*{hej}


\end{document}