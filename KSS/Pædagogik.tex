\documentclass[12pt, letterpaper]{article}
\usepackage[utf8]{inputenc}
\usepackage{graphicx}
\graphicspath{ {C:/Users/Martin/Pictures/} }
\usepackage{hyperref}




\begin{document}

\begin{titlepage}

\newcommand{\HRule}{\rule{\linewidth}{0.5mm}} % Defines a new command for the horizontal lines, change thickness here

\center % Center everything on the page
 
%----------------------------------------------------------------------------------------
%	HEADING SECTIONS
%----------------------------------------------------------------------------------------

\textsc{\LARGE Aarhus universitet}\\[1.5cm] % Name of your university/college
\textsc{\Large Pædagogik}\\[0.5cm] % Major heading such as course name
\textsc{\large Semester 3}\\[0.5cm] % Minor heading such as course title

%----------------------------------------------------------------------------------------
%	TITLE SECTION
%----------------------------------------------------------------------------------------

\HRule \\[0.4cm]
{ \huge \bfseries NOTER}\\[0.4cm] % Title of your document
\HRule \\[1.5cm]
 
%----------------------------------------------------------------------------------------
%	AUTHOR SECTION
%----------------------------------------------------------------------------------------

% If you don't want a supervisor, uncomment the two lines below and remove the section above
\Large \emph{Studerende:}\\[1cm]
Mette \textsc{Hammer Nielsen-Kudsk}\\[0,5cm] % Your name
Martin \textsc{Banasik}\\[1cm] % Your name
%----------------------------------------------------------------------------------------
%	DATE SECTION
%----------------------------------------------------------------------------------------

{\large \today}\\[1,2cm] % Date, change the \today to a set date if you want to be precise

%----------------------------------------------------------------------------------------
%	LOGO SECTION
%----------------------------------------------------------------------------------------

\includegraphics[scale=0.5]{au}\\ % Include a department/university logo - this will require the graphicx package
 
%----------------------------------------------------------------------------------------

\vfill % Fill the rest of the page with whitespace
\end{titlepage}
\newpage



\section*{Intro til faget pædagogik}

Faget drejer sig om:
\begin{itemize}
\item \textsc {Læring}: personlig forandrings- og udviklingsproces.

\item \textsc {Undervisning}: tilrettelagte læringsaktiviteter

\item \textsc {Didaktik}: Planlægning af undervisning
\end{itemize}


Når man taler om \textsc {Læring}, så taler man ikke om undervisning. Hvordan lærer man noget? Hvordan forandre man sig? Hvordan udvikler man sig, sine kompetencer. Læring kan sættes på en skabelon. Nervesystemet modens igennem barndommen, bliver dygtigere. Ved stimulation udvikles man.\newline

\textsc {Undervisning} skal være tilrettelagt. 
Man bruger undervisning når der skal læres noget bestemt.\newline

\textsc {Didaktik} er om hvordan lærer mennesker. Hvordan kan man undervise andre mennesker. Hvordan planlægger man sin undervisning, 360 grader så man ikke har glemt noget.

\begin{center}
\includegraphics[width=\textwidth]{billede1}
\end{center}


\textsc {Information}: at informerer er ikke undervisning. Når man informere forventer man ikke noget(envejs kommunikation) ex. "Så har I hørt det!" Der er meget der kan gå galt ved opfattelsen af den sendte information. Det er Ikke det samme som at man har overført en bestemt viden til et andet menneske. Hvad skal der til for at information bliver til noget lært? 
Det informeret skal blive til noget personligt som man selv optager(selv kunne give videre) laver ens egen konstruktion (ejerskab). Det sker ved dialog, sat viden på spil sammen med andre eller ved et stykke arbejde. Læseferie: tvunget til at forholde sig til viden. Hvad skal jeg kunne for at bestå. Bliver først rigtig virkelig og ens egen personlige viden ved at afprøve den i den virkelige verden. \newline


\textsc {Vejledning}: 
Bliver lidt brugt som undervisning, men er mere som coaching, mentor, mesterlære. Formålet er at sætte en udvikling i gang, men har måske ikke et bestemt mål, men det mål vil personen selv kunne finde ud af i løbet af processen. \newline 



\textsc {Undervisning}:
Intentionen om at der skal ske noget mellem underviser og elev. Man har et bestemt mål om undervisningen. 


Læring: hvor der er hul i vedkommens kompetencer. 

\begin{center}
\includegraphics[width=\textwidth]{billede2}
\end{center}

Hvordan kan man planlægge undervisningen? Skal igennem nogle analyser inden man kan planlægge undervisning. Pæd scenarie: Hvad, hvorfor, hvordan. Så ved deltagerne deres rolle og hvorfor.


Kan du lære andre noget?

Vi lærer selv. Stimulerer og forstyrre.  Det bliver til læring. Undervisningen skal være varieret.
Hvem kan vi komme til at undervise:\newline
"kodere" $\leftrightarrow$ Slutbrugere:\newline
			Fagprofessionelle  i sundhedsvæsnet
			Pårørende
			Patienter/borgere \newline

Det sker også igennem Dialog, samarbejde, afprøvning


\newpage

\section*{Behavioristen} 


Bruges af Psykologer.
Det handler om Adfærd. Hvordan ændre man adfærd. Man ved ikke hvad der sker inde i hovedet (blackboks), så man dokumenterer deres adfærd ud af til, ved brug af bestemte stimuli.

Hvordan reagerer man ved forskellige stimuli. Teorier blev udviklet. Prøvet først med dyr og så gik man over til mennesker.

Man kan lave en adfærdsændring ved at ændre adfærden ved bestemte stimuli.  

Adfærdsterapi. Problemer. Dem der skal gøre noget forstår ikke hvorfor de gør det. Effektivt og fungerer ubemærket.

Hvis man bliver anerkendt og rost, så bliver man ved med at gøre det. Der stimuleres til at gøre det godt igen næste gang. Behageligheder.

Motivationen/adfærd falder hvis man ikke får sine stimuli i hverdagen.


Hvis en som står svagt der kan de skubbes i den rigtige retning og yderligere styrke deres læring.

Hvordan ændre man en adfærd. Igennem dialog. Taler til en fornuftig løsning. "Du kan bruge denne teknologi til at øge livs kvalitet". Vi skal have fat i fornuften. Den gode dialog. Begynd at arbejde med motivationen. "Gå uden om jeg kan ikke teknologi". Til at starte med at der ikke stilles krav til nogle læring. 
Være anerkendende og ikke vise tegn på at man virker opgivende, men se et potentiale i vedkommende.

Hvis de ikke vil have hjælp. Lave en forståelses afklaring. Evner. Tale til fornuften. "For en sikkerheds skyld, gennemgår vi hvordan du vil håndtere denne teknologi". Hvis der skulle være nogle huller i deres viden eller kunnen. 


\begin{center}
\includegraphics[width=\textwidth]{billede3}
\end{center}

Hvis den lærende er motiveret og undervisningen er tilpasset og lærende har kapacitet. Det foregår hos den lærte.

\begin{center}
\includegraphics[width=\textwidth]{billede4}
\end{center}

\begin{center}
\includegraphics[width=\textwidth]{billede5}
\end{center}

Hvorfor savler hunde. Bliver overført til at sammenligne med hvad der kunne gøres i skole situationer. 

\begin{center}
\includegraphics[width=\textwidth]{billede6}
\end{center}

\begin{center}
\includegraphics[width=\textwidth]{billede7}
\end{center}

Giver en øgede læring.

\begin{center}
\includegraphics[width=\textwidth]{billede8}
\end{center}

Medlæring: konsekvens
Kommer til at møde slutbrugere der siger at de ikke ved noget om teknologi. De har haft oplevelser, de har haft negativ respons på. Eller fået af vide de er en ITspasser. Resultat: Dette medfører at teknologi er ikke lige mig. De har allerede lukket af. Der sker ingen kommunikation. "Han kan alligevel ikke lære mig noget". Der kan være indgroet forestillinger. Kan også møde det modsatte, hvor der har været for mange positive anerkendelse. 

Spørgsmål til at starte med.
Hvordan har du det med teknologi? Hvordan skal vi starte med det her? Hvordan har du det med computer?
Den positive og negative respons er afgørende for hvordan lærings processen vil blive.


\begin{center}
\includegraphics[width=\textwidth]{billede9}
\end{center}

Har arbejdes der med grupper med folk med få prævigilier som står svagt. Hvordan kan de klare sig bedre/hjælpes?

Ved brug af mentorer som de kan spejles i.

Ved undervisningen er vi coaches, ved fremtoning, kan de hjælpes på vej. Man viser hvordan man gør det rigtige. Ex. En konflikt. Når noget ikke virke som det skal. Skal man tager det rolig med ens egen adfærd.

Hvad betyder mentor, undervisers adfærd. Går systematisk frem og tager det roligt.

Man overtager/kopirer andres måde at reagere på og til sidst gør man det selv automatisk.

\begin{center}
\includegraphics[width=\textwidth]{billede10}
\end{center}

\begin{center}
\includegraphics[width=\textwidth]{billede11}
\end{center}


Læringen foregår inde i hovedet. Jeg har lavet en konstruktion. Man tror kun at der findes konstruktioner ved konstruktivisme. 
Vi mennesker er aldrig alene. Vi har dialog og stimuli. 
Samspil mellem underviser og elev. Det omgivende samfund spiller også ind i hvordan læringsprocessen skal foregår. 

\begin{center}
\includegraphics[width=\textwidth]{billede12}
\end{center}

En universel læring.
Forskellige dimensioner der betyder for den enkeltes læring. 
Motivation er en enorm faktor for læring. Giver energi til det der skal arbejdes og læres.
Samspil på holdet betyder hvor meget vi lærer.
Indhold at der gives opgaver. Man vil ikke kunne dokumenterer objektivt. Hvad betyder skolen i forhold til læringen.
Drivkraft(det er psykologerne som arbejder med dette.)

\begin{center}
\includegraphics[width=\textwidth]{billede13}
\end{center}

\begin{center}
\includegraphics[width=\textwidth]{billede14}
\end{center}
Kumulativ læring. Mekanisk læring. Hvis vi skal lære en kode som vi ikke kan referere til, en pin kode, men koden kan ikke sættes sammen med noget. Det giver ingen mening. 

\begin{center}
\includegraphics[width=\textwidth]{billede15}
\end{center}

Assimilativ er læring hvor man bygger op og føjer til ens viden. Giver mening til noget man allerede ved. Vi starter med basis og lægger på hele tiden. Den studerende føler at de tilføjer læring på det de allerede ved. Derfor er det vigtig at vide hvad de allerede ved. Kender deres niveau, så man kan starte med undervisningen der. Hvor er deres nærmeste zone for så at starte der.

\begin{center}
\includegraphics[width=\textwidth]{billede16}
\end{center}

Akkomodativ læring 
Når lærings indhold ikke passer til den viden og niveau til den underviste.

En gruppe af sygeplejesker. KAN HAVE MEGET VARIERET VIDEN OM TEKNOLOGI ELLER sundhedsviden. Situationen kan være at de bliver nødt til at slettet den gamle viden, for at forstå den nye viden. Det jeg skal lære nu kan ikke passe ind med det jeg allerede ved. Holdninger passer ikke. Først nedbryde og byg op igen. 

Et ex. Patienten der sidder ved lægen - har for højt blodtryk. Vedkommen får at vide at forbruget af fedt skal ned sættes. OK - siger patienten - jeg har ikke spist fedt i mange år, jeg bruger kun smør på rugbrødet. Der skal afklares. Hvad er fedt? Det bedstefar spiste på brød. Med fedt menes der osv. Og det er ikke så godt for helbredet. Men der er fedt som er godt - det skal du spise i stedet for.
Borgerne skal reset hvad han nu skal spise - med den nye viden om kost. Det er besværligt og anstrengende så man til sidst kan leve et liv at man ikke længere skal have for meget fedt.

Slutbrugeren har en forestilling - nej det betyder ikke sådan- det har en anden betydning. Der skal konstrueres en ny viden og hverdag.
Det vil senere blive til vaner og integreret i ens liv.

\begin{center}
\includegraphics[width=\textwidth]{billede17}
\end{center}

Transformativ læring
Når man skifter identiteten. Pga. højskole, sygedom. Lærer noget nyt. Total ændret holdninger. Lever livet på en ny måde.

Måske vil den teknologi vi kommer med kan den være med til at ændre forestillingen om hvad man kan. Ex. Mere social kontakt, fysisk aktiv, som så kan give et nyt liv og en øgede livskvalitet.

Ved en identitets ændring vil kræve en akkumulativ læring inden eller er det ikke muligt at skabe en identitets ændring. 
Ændre sig for at være offer til at tænke at kan gøre nogle ting selv og være aktiv.


\begin{center}
\includegraphics[width=\textwidth]{billede18}
\end{center}

Det er ikke mit koncept. Jeg har ikke brug for denne læring. Det er ikke min motivation. Det er ikke min ide. Man skubber det væk. Tilpasser indholdet til det man allerede har. 
Fra hverdagen. Faste tanker og svar. Hvordan livet ses og opleves. De ændre sig ikke. Skubber besværlige ting væk. Man gider ikke og kan ikke overskue det.

\begin{center}
\includegraphics[width=\textwidth]{billede19}
\end{center}

I et sammenhæng hvor man ikke synes at man føler sig hjemme. Det må de andre godt forholde sig til man det er ikke en del af mit livskoncept.


\begin{center}
\includegraphics[width=\textwidth]{billede20}
\end{center}


Hvis man hjælper den lærende med at vide hvad de skal kunne - hvor er hullet mellem hvad de ved og hvad de skal kunne vide. Det er en god grobund. 
Hvis den lærende tror at det ikke er noget som er nødvendigt så bliver det akkomodativ læring. 
Hvad kan i allerede? Hvor er i henne i jeres liv? Hvad kan vi bibringer for at give jer en fordel og hjælpe.

Stille de rigtige spørgsmål så vi bliver klogere på hinanden. Åbne spørgsmål (lukket er ja/nej spørgsmål).

Tilpasse undervisning til hvor de er nu. Så lidt akkomodativ læring og mest Assimilativ (tilføjer viden).

En film om forudsætninger og motivation.\newline

\url{http://www.youtube.com/watch?v=EzI80kyiUB4}















\end{document}
